\documentclass[a4paper, 12pt]{article}
\usepackage[UTF8]{ctex}
\usepackage{graphicx} % for pictures
\usepackage[margin=2cm]{geometry}
\usepackage{float} % for pictures
\usepackage{xeCJK}% for chinese input
\usepackage{setspace} % for line spacing


%%% font settings
\setCJKmainfont{SimSun}
\setmainfont{Times New Roman}
\newCJKfontfamily[mingliu]\mlu{MingLiU}

%%% format settings
% 要注意,latex的字号变动会覆盖baselineskip
% latex默认的baselineskip是fontsize的1.2倍,所以改了fontsize之后baselineskip也变了
% 行距计算:baselineskip * baselinestretch,baselinestretch和linespread其实是差不多的,只是取值不一样,后者去1.3对应前者的1.5,1.6对应2;所以一般设置行距直接改后面的因子

%%% new commands
\newcommand{\mainheader}{东南大学 2019 届本科生毕业设计(论文)}


\begin{document}

\begin{figure}[H]
\centering
\includegraphics[width=4.28in]{seu}
\end{figure}
\vspace{26pt}

\begin{center}
\setstretch{3.2}
\Large
\textbf{题目} \quad \underline{\makebox[250pt][c]{\textbf{图像去雾化方法研究}}}\\
\underline{\makebox[80pt][c]{\textbf{交通学院}}}院(系)\quad \underline{\makebox[80pt][c]{\bf 交通工程}}专业 \\
学\qquad 号 \underline{\makebox[300pt][c]{\textbf{21015111}}}\\
学生姓名 \underline{\makebox[300pt][c]{\textbf{周冬秦}}}\\
指导教师 \underline{\makebox[300pt][c]{\textbf{何铁军 教授}}}\\
起止日期 \underline{\makebox[300pt][c]{\textbf{2019年2月25日至2019年6月2日}}}\\
设计地点 \underline{\makebox[300pt][c]{\textbf{东南大学交通学院203室}}}

\end{center}
\newpage








\textbf{fuck}

\bigskip
你好啊啊

{\heiti 你好啊啊,黑体}

{\kaishu 你好啊啊,楷书}

{\mlu 你好啊啊,细明体}

what the hell is going on 































\end{document}