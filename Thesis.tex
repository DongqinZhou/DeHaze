\documentclass[a4paper, 12pt, oneside]{report}
\usepackage[UTF8]{ctex}
\usepackage{graphicx} % for pictures
\usepackage{subcaption} % for caption in multi-figures
%\usepackage{subfig}
\usepackage[left = 2.5cm, headsep = 0.5cm, headheight = 0.5cm, top = 2cm, bottom = 2cm, right = 2cm]{geometry}
\usepackage{tocloft} % for tableofcontents
\usepackage{float} % for pictures
\usepackage{xeCJK}% for chinese input
\usepackage{setspace} % for line spacing
\usepackage{anyfontsize}% set font size
\usepackage{titlesec} % set chapter style
\usepackage{CJKnumb} % for chinese number
\usepackage{titletoc} % title in TOC
\usepackage{amsmath} % for mathematical expressions, such as matrix
\usepackage{caption} % captipon setting
\usepackage{multirow} % for multirow in tables
\usepackage{hyperref} % for hyperlink
\usepackage[nottoc]{tocbibind} % add bibliography to toc
\usepackage{fancyhdr} % for header and footer
\usepackage{etoolbox}
\usepackage{xcolor}
\patchcmd{\chapter}{\thispagestyle{plain}}{\thispagestyle{fancy}}{}{} %reset chapter page style



%%% font settings
\setCJKmainfont{SimSun}
\setmainfont{Times New Roman}
%\newCJKfontfamily[mingliu]\mlu{MingLiU}

\captionsetup{labelsep=space} % remove colon in caption
\captionsetup[table]{labelsep=space}

%%% format settings
% In default, changing font size will change line skip, which is set to 1.2 * fontsize. 
% Line spacing = baselineskip * baselinestretch or Line spacing = baselineskip * linespread, while baselinestretch and linespread differ only in value.

%%% new commands
\newcommand{\mainheader}{\fontsize{9}{10} \selectfont 东南大学 2019 届本科生毕业设计(论文)}
\renewcommand{\contentsname}{\centerline{ \fontsize{16}{19.2} \selectfont \heiti 目 \qquad 录 } } % not recommended, but could do the trick.
\renewcommand{\cftchapleader}{\cftdotfill{\cftdotsep}}
\renewcommand{\cftdot}{$\cdot$}
\renewcommand{\cftdotsep}{1}
\setlength{\cftbeforechapskip}{1em}
\setlength{\cftbeforetoctitleskip}{-2em} % adjust the distance from top of page to toc title
\newcommand{\acknowledgementtitle}{\heiti \fontsize{14}{17} \selectfont 致\quad 谢}
\newcommand{\acknowledgementtitletoc}{致谢}
\renewcommand{\arraystretch}{1.5} % table stretch
\renewcommand{\labelenumi}{\roman{enumi}) } % change enumeratior index
%\renewcommand{\bibname}{\protect\leftline{\heiti \fontsize{14}{17} \selectfont 参考文献}} % This changes the format of bibliography title in TOC as well. But because it's easier to use, I leave it here. And the TOC title could be modified manually in .toc file (dirty trick though).

\renewcommand{\bibname}{\protect\leftline{参考文献}}
%\setcounter{secnumdepth}{3}% Number \subsubsection https://tex.stackexchange.com/questions/146304/subsubsection-in-a-report-style-document/146307 Very good information about the toc and numbering for reports
%\setcounter{tocdepth}{3}


%%% Chinese abstract
\renewenvironment{abstract}[1]
{
\newcommand{\keywords}{#1}
\phantomsection
\addcontentsline{toc}{chapter}{摘要\quad }
\vspace*{12pt}
\begin{center}
\fontsize{18}{21.6}\selectfont 摘\quad 要
\end{center}
\fontsize{12}{18}\selectfont
}
{
  \\[1\baselineskip]
  关键词: \keywords

}

%%% English abstract
\newenvironment{englishabstract}[1]
{
\newcommand{\keywords}{#1}
\phantomsection
\addcontentsline{toc}{chapter}{Abstract\quad }
\vspace*{12pt}
\begin{center}
ABSTRACT
\end{center}
\fontsize{12}{18}\selectfont
}
{
\\[1\baselineskip]
KEY WORDS: \keywords

}

%%% chapter format 
\titleformat
{\chapter} % command
{\centering \heiti \fontsize{15}{18} \selectfont} % format
{第\CJKnumber{\thechapter}章} % label
{1ex} % sep
{\centering} % before-code
\titlespacing{\chapter}{0pt}{12pt}{24pt} % set chapter title spacing

%%% section format
\titleformat
{\section} % command
{\heiti \fontsize{14}{17} \selectfont} % format
{\thesection \ } % label
{0.1ex} % sep
{} % before-code

%%% subsection format
\titleformat
{\subsection} % command
{\fontsize{12}{14} \selectfont \bfseries } % format
{\thesubsection \ } % label
{0.1ex} % sep
{
\bfseries
} % before-code


%%% chapter style in TOC
\newcommand{\titlecontentschapter}{%
\titlecontents{chapter}[0pt]{\vspace{1\baselineskip}\normalfont} % determine the spacing for chapters in TOC
{第\CJKnumber{\thecontentslabel}章\quad}{}
{\hspace{.5em}\titlerule*[5pt]{$\cdot$}\contentspage}
}

%%% Acknowledgement
\newenvironment{Acknowledgement}[1][\acknowledgementtitle]
{%
  
  \phantomsection
  \addcontentsline{toc}{chapter}{\acknowledgementtitletoc}
  \chapter*{#1}

}


\begin{document}

%%% Cover
\vspace*{24pt}
\thispagestyle{empty}
\begin{figure}[H]
\centering
\includegraphics[width=4.28in, height = 1.53in]{seu}
\end{figure}
%\vspace{16pt}

\begin{center}
{\fontsize{16}{46}\selectfont % first : fontsize; second : baselineskip

\textbf{题目} \quad \underline{\makebox[250pt][c]{\textbf{THESIS TITLE}}}\\
\underline{\makebox[100pt][c]{\bf DEPARTMENT}}院(系)\quad \underline{\makebox[100pt][c]{\bf SPECIALIZATION}}专业 \\
学\qquad 号 \underline{\makebox[300pt][c]{\textbf{STUDENT NUMBER}}}\\
学生姓名 \underline{\makebox[300pt][c]{\textbf{NAME}}}\\
指导教师 \underline{\makebox[300pt][c]{\textbf{NAME \& TITLE OF SUPERVISOR}}}\\
起止日期 \underline{\makebox[300pt][c]{\textbf{TIME}}}\\
设计地点 \underline{\makebox[300pt][c]{\textbf{LOCATION}}}\\
}
\end{center}
\newpage

%%% 独创性声明、授权声明
\thispagestyle{empty}
\vspace*{36pt}
\begin{center}
\textbf{\fontsize{16}{19.2} \selectfont 东南大学毕业(设计)论文 \\ \medskip 独创性声明}
\end{center}
\par 本人声明所呈交的毕业(设计)论文是我个人在导师指导下进行的研究工作及取得的研究成果。尽我所知,除了文中特别加以标注和致谢的地方外,论文中不包含其他人已经发表或撰写过的研究成果,也不包含为获得东南大学或其它教育机构的学位或证书而使用过的材料。与我一同工作的同志对本研究所做的任何贡献均已在论文中作了明确的说明并表示了谢意。
\begin{flushright}
论文作者签名:\rule{2cm}{0.05em} \quad 日期:\rule{2cm}{0.05em}
\end{flushright}
\vspace{4cm}

\begin{center}
\textbf{\fontsize{16}{19.2} \selectfont 东南大学毕业(设计)论文使用 \\ \medskip 授权声明}
\end{center}
\par 东南大学有权保留本人所送交毕业(设计)论文的复印件和电子文档,可以采用影印、缩印或其他复制手段保存论文。本人电子文档的内容和纸质论文的内容相一致。除在保密期内的保密论文和在技术保护期限内的论文外,允许论文被查阅和借阅,可以公布(包括以电子信息形式刊登)论文的全部或中、英文摘要等部分内容。论文的公布(包括以电子信息形式刊登)授权东南大学教务处办理。
\begin{flushright}
论文作者签名:\rule{2cm}{0.05em} \quad 导师签名:\rule{2cm}{0.05em} \quad 日期:\rule{2cm}{0.05em}
\end{flushright}
\newpage

%%% Abstracts
\pagenumbering{Roman}
%{\fontsize{12}{18}\selectfont
\begin{abstract}{KEYWORDS IN CHINESE}
\vspace{12pt}
\par ENTER CHINESE ABSTRACT HERE.
\end{abstract}

\newpage

\begin{englishabstract}{KEYWORDS IN ENGLISH}
\vspace{12pt}
\par ENTER ENGLISH ABSTRACT HERE.  
\end{englishabstract}

\newpage

%%% TOC 
% good info for removing the footer on TOC: https://tex.stackexchange.com/questions/180874/remove-page-numbers-from-footer-for-multi-page-table-of-contents/180877
\pagestyle{empty}
\begingroup
\renewcommand{\leftline}{}
\renewcommand{\thispagestyle}[1]{}
\fontsize{12pt}{18pt} \selectfont
\tableofcontents
\endgroup
\newpage

%%% Mainmatter
\pagestyle{fancy}
\fancyfoot{}
\fancyhead{}
\fancyhead[C]{\mainheader}
\fancyhead[R]{\fontsize{9}{10}\selectfont 第 \thepage 页}
\setlength{\footskip}{0.75cm}
{\fontsize{12}{18}\selectfont % set line spacing for main text
\pagenumbering{arabic}
\titlecontentschapter   % 在这里才启用前面定义的目录章节格式,不在前面启用,因为摘要也算作章节,牛逼!
\chapter{绪论\quad}
\section{研究背景和意义\quad}

CITATION EXAMPLE\cite{ref1}。

EQUATION EXAMPLE
\begin{equation} \label{eq:1.1}
{\bf I(x)} = {\bf J(x)} t{\bf (x)} + {\bf A}(1 - t{\bf (x)})
\end{equation}


\section{研究现状\quad}
CITE EQUATION(\ref{eq:1.1}) 

\subsection{SUBSECTION\quad}

\section{研究目的和研究内容\quad}


\chapter{CHAPTER 2\quad}


} % ends line spacing setting here

\begin{Acknowledgement}{}
\par To my family, friends, and tearchers.

\end{Acknowledgement}

\begin{thebibliography}{99}  
\fontsize{10.5}{10.5} \selectfont

\setlength{\itemsep}{0pt}
\bibitem{ref1} https://github.com/DongqinZhou/DeHaze.

\end{thebibliography}



\end{document}